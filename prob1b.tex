% Problem 1b goes below

For the first table $Student$, the primary key is $SID$, which can get you a student's name and email. Although $SID$ and $Email$ are both unique to each student, $SID$ was chosen as the key because it's used more in relation to the other attributes we're exploring, and in other tables, each student will be referred to by their $SID$.

The table $Course$ is course information of each course enrollment. The primary key is course ID ($CID$) and $Term$ number. From that, we can get the full course name including $Subject$, $CSRE$, $UNITS$ offered, and $Section$ number. We noticed that each section has a unique $CID$ in a $Term$ even if the class is the same, but $CID$ can be reused in different terms.

Each $Course$ has at least one $Meeting$. Depending on $Meeting Type$, we can find the $Days$, $Time$, $Building$, $Room$, and $Instructor$ for that meeting.

Each entry in $Enrollment$ is an indication of a student's status in the term they were enrolled in a certain course. If we know the student's $SID$ and the $CID$ and $Term$ for the course they're taking and when, we can find what $Grade$ they got, how many $Units$ they took for that class, the $Course Seat$ they were in for that class, their registration $Status$, $Class$ standing, and their $Level$. Since students can change major, each enrollment keeps track of what $Major$ the student currently had declared at the time.

$Grade$ is determined on a numeric scale and has an assigned $Number Grade$ according to their letter grade.